%!TEX root = eyz_assignment4.tex
\textbf{Problem 1:} Consider a Gaussian linear dynamical system (LDS),
\begin{align*}
    p(x_{1:T}, y_{1:T}) &= \cN(x_1 \mid 0, q^2) \left[ \prod_{t=2}^T \cN(x_t \mid a x_{t-1} + b, q^2) \right] 
    \left[ \prod_{t=1}^T \cN(y_{t} \mid x_t, r^2) \right],
\end{align*}
for~$x_t, y_t \in \reals$ for all~$t$, and parameters~$a, b \in \reals$ and $q^2, r^2 \in \reals_+$.  
Compute the forward filtered distribution~$p(x_t \mid y_{1:t})$ in terms of the model parameters 
and the filtered distribution~$p(x_{t-1} \mid y_{1:t-1})$.  Solve for the base case~$p(x_1 \mid y_1)$.  
For reference, consult the state space modeling chapters of either the Bishop or the Murphy textbook.

\begin{solution}

A directed graph of linear Gaussians is equivalent to a joint Gaussian distribution over all of variables,
so the marginal distribution of each variable is also Gaussian.
Similarly, in this linear Gaussian dynamical system, the posterior marginals of latent variables will be Gaussian.
So, we know that the forward filtered distributions take the form
\begin{align*}
	p(x_t \mid y_{1:t}) &= \cN(x_t \given m_t, v_t) \\
	p(x_{t-1} \mid y_{1:t-1}) &= \cN(x_{t-1} \given m_{t-1}, v_{t-1})
\end{align*}

We will first derive the forward filtered distribution $p(x_t \mid y_{1:t})$ by evaluating the posterior marginal and conditional distributions of a linear Gaussian model (method A), and then confirm with the derived Kalman filter equations (method B).

\textbf{Method A} \\
Given the model parameters and the filtered distribution from the previous timestep $p(x_{t-1} \mid y_{1:t-1}) \propto \cN(x_{t-1}\given m_{t-1}, v_{t-1})$, we have
\begin{align*}
	p(x_t \mid y_{1:t})
		&\propto p(y_{1:t} \given x_t)\, p(x_t) \\
		&\propto p(y_{t} \given x_t)\, p(y_{1:t-1}\given x_t)\, p(x_t) \\
		&\propto p(y_{t} \given x_t)\, \int p(y_{1:t-1}\given x_t, x_{t-1})\, p(x_t \given x_{t-1})\, \textrm{d}x_{t-1} \\
		&\propto p(y_{t} \given x_t)\, \int p(x_{t-1} \given y_{1:t-1})\, p(x_t \given x_{t-1})\, \textrm{d}x_{t-1} \\
		&\propto \cN(y_t \given x_t, r^2)\,
			\int \cN(x_{t-1}\given m_{t-1}, v_{t-1})\,
			\cN(x_t \given ax_{t-1} +b, q^2)\, \textrm{d}x_{t-1}.
		\numberthis\label{q1_part1end}
\end{align*}
Using the posterior marginal distribution results for linear Gaussian models (Bishop 2.115), we can simplify the integral as follows
\begin{align*}
	p(x_t)
		&= \int p(x_{t-1}) \, p(x_t \given x_{t-1})\, \textrm{d}x_{t-1} \\
		&= \int \cN(x_{t-1}\given m_{t-1}, V_{t-1})\,
			\cN(x_t \given ax_{t-1} +b, q^2)\, \textrm{d}x_{t-1} \\
		&= \int \cN(x_t \given a m_{t-1}, p_{t-1})\, \textrm{d}x_{t-1}
		= \cN(x_t \given a m_{t-1}, p_{t-1})
\end{align*}
where $p_{t-1} = q^2 + a^2 v_{t-1}$. Then, continuing from Eqn.~\eqref{q1_part1end}, we again make use of Bishop Eqns. (2.115) and (2.116) to identify,

\begin{align*}
	\textrm{\textcolor{gray}{Bishop Eqns. 2.113-2.117}}
		&\qquad \textrm{This problem} \\
	\textcolor{gray}{p(x) = \cN(x \given \tilde{\mu}, \Lambda^\inv)}
		&\qquad p(x_t) = \cN(x_t \given a\cdot m_{t-1}, p_{t-1})\\
	\textcolor{gray}{p(y \given x)
		= \cN(y \given \tilde{A}x + \tilde{b}, L^\inv)}
		&\qquad p(y_t \given x_t) = \cN(y_t \given x_t, r^2) \\
	\textcolor{gray}{p(y)
		= \cN(y \given \tilde{A}\,\mu + b,
		\tilde{L}^\inv + \tilde{A}\Lambda \tilde{A}^\trans)}
		&\qquad p(y_t) = \cN(y_t \given a\cdot m_{t-1}, r^2 + p_{t-1}^\inv)\\
	\textcolor{gray}{p(x \given y) = 
		\cN(x \given \tilde{\Sigma}\,(\tilde{A}^\trans L (y-\tilde{b}) + \Lambda \tilde{\mu}), \tilde{\Sigma})}
		&\qquad p(x_t \given y_t) =
			\cN(x_t \given \tilde{\Sigma}\,(r^{-2}y + p_{t-1}^\inv(a\cdot m_{t-1})), \tilde{\Sigma})\\
	\textcolor{gray}{\quad \textrm{for } \tilde{\Sigma} = (\Lambda + A^\trans L A)^\inv}
		&\qquad \quad \textrm{for }
			\tilde{\Sigma} = (p_{t-1}^\inv + r^{-2})^\inv
\end{align*}
where $\textcolor{gray}{\tilde{\mu}=}a\cdot m_{t-1}$,
$\textcolor{gray}{\Lambda^\inv=}p_{t-1}$,
$\textcolor{gray}{\tilde{A}=}1$,
$\textcolor{gray}{\tilde{b}=}0$, and
$\textcolor{gray}{L^\inv=}r^2$.
So $p(x_t \mid y_{1:t}) = \cN(x_t \given m_t, v_t)$ for

\begin{align*}
	m_t &= \frac{p_{t-1}\, y_t + a\,r^2\,m_{t-1}}{r^2 + p_{t-1}}
	\numberthis\label{q1_solution_a1} \\
	v_t &= \frac{r^2\,p_{t-1}}{r^2 + p_{t-1}}
	\numberthis\label{q1_solution_a2}
\end{align*}

\textbf{Method B}
\begin{align*}
	\textrm{\textcolor{gray}{Bishop LDS Model}}
		&\qquad \textrm{This problem} \\
	\textcolor{gray}{p(x_n \given x_{n-1}) = \cN(x_n \given A x_{n-1}, \Gamma)}
		&\qquad p(x_t \given x_{t-1}) = \cN(x_n \given a x_{t-1}, q^2) \\
	\textcolor{gray}{p(y_n \given x_n) = \cN(x_t \given C x_{n}, \Sigma)}
		&\qquad p(y_t \given x_t) = \cN(y_t \given x_{t}, r^2) \\
	\\
	\textrm{\textcolor{gray}{Bishop Eqns. 13.89-13.92}}
		&\qquad \textrm{This problem} \\
	\textcolor{gray}{\mu_n = A \mu_{n-1} + K_n(x_n -CA\mu_{n-1})}
		&\qquad m_t = a\cdot m_{t-1} + k_t(y_t - a\cdot m_{t-1}) \\
	\textcolor{gray}{V_n = (I - K_n C)\, P_{n-1}}
		&\qquad v_t = (1 - k_t)\,p_{t-1}\\
	\textcolor{gray}{c_n = \cN(x_n \given CA\mu_{n-1}, CP_{n-1}C^\trans + \Sigma)}
		&\qquad p(y_t) = \cN(y_t \given a\cdot m_{t-1}, p_{t-1} + r^2)\\
	\textcolor{gray}{K_n = P_{n-1}\,C^\trans (CP_{n-1}C^\trans + \Sigma)^\inv}
		&\qquad k_t = p_{t-1} (p_{t-1} + r^2)^\inv \\
\end{align*}
If we expand out the expressions for $k_t$, $m_t$, and $v_t$, we find that they agree with Eqn.~\eqref{q1_solution_a1}~and~\eqref{q1_solution_a2} which were found via the first method:
\begin{align*}
	k_t &= \frac{p_{t-1}}{p_{t-1}+r^2} \\
	m_t &= a\cdot m_{t-1} + k_t(y_t - a\cdot m_{t-1}) \\
		&= \frac{a\cdot m_{t-1}\,(p_{t-1}+r^2)}{p_{t-1}+r^2}
			+ \frac{p_{t-1}\,(y_t - a\cdot m_{t-1})}{p_{t-1}+r^2} \\
		&= \frac{a\,p_{t-1}\,m_{t-1} + a\,r^2\,m_{t-1} +
				p_{t-1}\,y_t - a\,p_{t-1}\,m_{t-1})}
			{p_{t-1}+r^2} \\
		&= \frac{a\,r^2\,m_{t-1} +p_{t-1}\,y_t} {p_{t-1}+r^2}
		\numberthis\label{q1_solution_b1} \\
	v_t &= (1-k_t)\,p_{t-1} \\
		&= \left(1-\frac{p_{t-1}}{p_{t-1}+r^2}\right) p_{t-1} \\
		&= \frac{p_{t-1}+r^2 - p_{t-1}}{p_{t-1}+r^2} p_{t-1}
			= \frac{p_{t-1}\,r^2 - p_{t-1}}{p_{t-1}+r^2}
			\numberthis\label{q1_solution_b2}
\end{align*}

The base case $p(x_1 \given y_1)$ is then given by Bishop Eqns.~(13.94-13.97):
\begin{align*}
	m_1 &= k_1\,y_1\\
	v_1 &= (1 - k_1)\, q^2\\
	c_1 &= \cN(y_1 \given 0, q^2 + r^2)\\
	k_1 &= \frac{q^2}{q^2 + r^2}
\end{align*}

In the Kalman filtering equations, we see how the new model mean $m_t$ is a weighted average of the evolved model mean $a\cdot m_{t-1}$ and the error betwen observation and evolved mean.
\end{solution}
